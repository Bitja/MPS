\section*{Introduction}
The primitive fight and flight response is stimulated in the sympathetic nervous system and is most prominent when perceiving a threat \cite{bodily_changes}. The body prepares for the exertion to deal with the stressful situation: Adrenalin and cortisol are released, and the body produces sweat in order to cool itself down. Stress has been investigated in video games, e.g. in first person shooters games that relies on quick responses. This study examines whether or not failure in a tablet game can stimulate a response similar to the fight, flight, or freeze response, as seen in a study by Cannon, (1927), also, the tend and befriend response defined by Taylor et al., (2000), which is a response to how women react in a stressful situation \cite{tendbefriend}. For this study we observed students’ reactions to a stress situation in a game. We designed a coding scheme of what we defined as tendencies in the four different types of responses. The coding scheme was used in the video analysis, focusing on what the subjects communicated to each other, their facial expressions, and how concentrated they were with the task of tracing an invisible line. 

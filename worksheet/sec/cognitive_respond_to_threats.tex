\section{Cognitive Respond to Threats}\label{cognition}
The cognitive response to a threat is the evaluation of the physiological changes as described in \ref{perception}. As we gain information from the brain and body because of how we perceive a threat, we start to evaluate the autonomic bodily changes to respond with the best possible reaction. Mainly we look at the two categories which are fight and flight, the oldest principle, dating back to 1927 \cite{bodily_changes}, described as reactions of anxiety and anger. An example where anxiety is created, would be a situation where one arrive at an important meeting, only to discover one is unprepared. This situation creates a sense of anxiety, where the heart rate rises together with blood pressure, respiration, and other autonomic bodily changes as a reaction to the perceived threat, inducing a flight reaction, fleeing the location where the situation occurred. If however, it was another person's fault one one was not prepared, there could be more biased towards an emotion of anger, wanting to fight, either a physically or verbally. \cite{bodily_changes}

\subsection{Changes in Cognitive Response}\label{cognitive_response}
The feeling of anxiety or anger, when put in a situation of perceived threat might change specific emotions, depending on the individual and their past memory of similar situations. An example of the cognitive process could be when a fire alarm goes of. The normal perception of the situation would be the fear of harm or threat. This initialises the cognitive process, which determines the reaction to the event depending on previous knowledge. If one is at home, one might have specific knowledge of what might have caused the threat, e.g. the burned lunch one forgot. Perhaps one would not flee, but rather fight the situation, by stopping the events setting off the alarm. On the other hand, does the alarm go off at the workplace, one could be more biased to flee, if one has no knowledge about the threat’s cause or location. These are also depending on the feelings of anger, as you just burned the food, or anxiety, as you do not know what set off the alarm at work.\cite{bodily_changes}

%\subsection{The Reactions According to Fight/Flight}
%As shortly exemplified in the above section \ref{cognitive_response}, the two main reactions referred to in literature when describing actions in a stressful situation, is often the fight/flight reactions. These are often seen as opposites, as the one is taken if situations creating anger, the fight, and the other in situations of anxiety. Emotions created from physiological changes bias the individual in either direction, which involves many of the autonomic bodily reactions from \ref{perception} but also other hormones comes into play, as the testosterone, oestrogen and dopamine. These reactions are thought as the two main actions taken in stressful situations, but others have started to show up in research, the freeze (tensing of muscles to unable to act to the situation), fright (becoming afraid), faint (automatic shut down to get away from the situation) and tend/befriend (social awareness reaction, most seen in women). We have decided to look more into the fight, flight, freeze and tend/befriend reaction, and described them further in the section below as motors to the cognitive response of a stressful situation 
%%\ref{motors}. 
%\cite{fightflight} \cite{5responses} \cite{tendbefriend}

%\cite{Bracha, H. S. Freeze, flight, fight, fright, faint: adaptationist perspectives on the acute stress response spectrum.} \cite{Biobehavioral Responses to Stress in Females: Tend-and-Befriend, not Fight-or-Flight}
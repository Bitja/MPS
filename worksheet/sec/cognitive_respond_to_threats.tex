\section{Cognitive Respond to Threats}
The main aspect of the cognitive aspect of responding to a threat, is the evaluation of the physiological changes as described in \ref{perception}. As we gain information from the brain and body because of how we perceive a threat, we start to evaluate the autonomic bodily changes to respond with the best possible reaction. Mainly we look at the two categories which are fight and flight, the oldest principle, dating back to 1929 \cite{bodily_changes}, described as reactions of anxiety and anger. An example where anxiety is created, would be a situation where you are late for a meeting or a class, where you experience every body else is getting ready for a test or presentation, which you yourself did not remember to prepare for. This situation creates a sense of anxiety, where the heart rate rises together with blood pressure, respiration, and other autonomic bodily changes as a reaction to the perceived threat, inducing a flight reaction, fleeing the location where the situation occured. On the other hand, if you were dependant on another colleague or student, and it were their fault you did not get the information about the test or presentation, you would be more biased towards an emotion of anger, wanting to fight, in either a physical or verbal action. \cite{bodily_changes}

\subsection{Changes in Cognitive Response}\label{cognitive_response}
The feeling of anxiety or anger when put in a situation of perceived threat, is dependant on the individual and their past memory of equal situations and their teaching of the reaction and changing of specific emotions. A great example of the cognitive process is a fire alarm situation. When a fire alarm goes of, the normal view of the situation would give you the 
perception of harm or threat. This initialises the cognitive process, which can decide the reaction to the event dependant on previous knowledge. If you for instance are at home, hearing the fire alarm, you would first evaluate the situation dependant on the people present. If you for instance are alone, the actions you have taken in the last couple of moments would influence your reaction, shower steam could set of the alarm, or burned food, and you would not flee but rather fight in the aspect of stopping the events setting off the alarm. On the other hand, does the alarm go off in the work place or university, you would be more biased to flee if it is the first time the alarm goes of, taking the action of flight. These are also dependant on the feelings of anger, as you just burned the food, or anxiety, as you do not know what set of the alarm at work.\cite{bodily_changes}

\subsection{The Reactions According to Fight/Flight}
As shortly exemplified in the above section \ref{cognitive_response}, the two main reactions referred to in literature when describing actions in a stressful situation, is often the fight/flight reactions. These are often seen as opposites, as the one is taken if situations creating anger, the fight, and the other in situations of anxiety. Emotions created from physiological changes bias the individual in either direction, which involves many of the autonomic bodily reactions from \ref{perception} but also other hormones comes into play, as the testosterone, oestrogen and dopamine. These reactions are thought as the two main actions taken in stressful situations, but others have started to show up in research, the freeze (tensing of muscles to unable to act to the situation), fright (becoming afraid), faint (automatic shut down to get away from the situation) and tend/befriend (social awareness reaction, most seen in women). We have decided to look more into the fight, flight, freeze and tend/befriend reaction, and described them further in the section below as motors to the cognitive response of a stressful situation 
%\ref{motors}. 
\cite{fightflight} \cite{5responses} \cite{tendbefriend}

%\cite{Bracha, H. S. Freeze, flight, fight, fright, faint: adaptationist perspectives on the acute stress response spectrum.} \cite{Biobehavioral Responses to Stress in Females: Tend-and-Befriend, not Fight-or-Flight}
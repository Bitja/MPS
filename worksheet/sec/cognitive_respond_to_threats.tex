\section{Cognitive Respond to Threats}

Evaluating the physiological changes (evaluating anxiety and aggression)
After the physiological changes in the body because of the perception of a situations, a reactions is created corresponding to fight/flight, and as further research has shown, freeze, fright, faint and tend/befriend. Even though we are not often in survival threatening situations, which is where the fight/flight reaction comes from, we are still reaction according to this in stressed situations, as someone attacking us, but also when we are taking a test we didn't prepare for, or another individual didn't make his part of a presentation you were working on together.

Cognition depends on the your perception and recognition of the situation, fire alarm example
The fight/flight is though also dependant on how we recognize the situation from a similar past event, as described in section \ref{perception}. For instance, if we hear a fire alarm, our first response is a threat to our survival, as we are near a fire. This information is automatic, but we still evaluate dependant on the current situation. The alarm could be have been started by a steam from a shower or because you were cooking, creating non harmful smoke and we would evaluate the situation as non-threatening becoming less stressed or anxious.

Could be dependant on the situation as the exam/presentation examples.
It could also depend on how the situation played out last time it happened. For instance if we are late for a class and we see everybody getting ready for a test you did not know you had, you would get anxious because you forgot to study, but if you know the teacher regularly give out random tests, or you are well known to be smart within the subject, you might not get that anxious, remembering the past event.
An example of aggression can be easily described as, if you are working in a pair group with a presentation, and the other person in the group did not do his part, you might be more biased towards aggression, as you believe it is not your fault, but if you forgot to tell him to do it, as he was sick the day the exercise was given, you might become anxious instead. %\cite{Bodily changes in pain, hunger, fear and rage}

Reaction according to fight/flight
The two main reactions we get from the physiological body reactions, is the anxious and aggressive states. These have been described of reactions of fight or flight, which in short is the anxious act of fleeing the scene, or the aggressive act of trying to solve the problem through anger. In newer research other reactions have started to show, the freeze (immobility or being passive), the fright (being afraid of the situation), faint (the bodily reaction of becoming unconscious), or the tend/befriend (trying to give social support before solving the situation).
These will all be further described in the following sections, with relation to how they would surface through stress induced gaming. %\cite{Bracha, H. S. Freeze, flight, fight, fright, faint: adaptationist perspectives on the acute stress response spectrum.} \cite{Biobehavioral Responses to Stress in Females: Tend-and-Befriend, not Fight-or-Flight}
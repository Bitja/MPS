\section{Perception of Threats} \label{perception}

Perceiving threats
	Evidence show that emotion affects behaviour either positively or negatively
	Emotions have possibly been created to identify threats and increase survival rates
	Preperation for situations affect our future reactions. The perception of a current situation is changed because of past corresponding situations, which we can relate the current to.
	Motor areas are affected by the emotional state.
	RT can be increased by threats away from you, and slowed by threats against you. Mutilation pictures induce a freeze like response. Freeze, and into flight, and if not possible to flee, fight.
	Threats against you will induce more physiological response than away from you.
	The physiological responses will change the perception, also of the threat and situations.
%	\cite{How you perceive threat determines your behavior}

Automatic nervous system
	ANS influences the function of the internal organs. regulates body functions as heart rate, breathing, blood pressure, digestion and others. On of the primary systems is the sympathetic nervous system, or sometimes referred to as the fight/flight system. 	%\cite{Pocock, Gillian (2006). Human Physiology (3rd ed.). Oxford University Press.}

Central command neurons
	ANS can be is directly related to stress, threats or other situations harmful to the individual.
	When a threat is perceived the amygdala reacts by processing emotions, memory and decision-making. It sends information to the hypatalamus which creates ACTH, creating cortisol (stress) and adrenaline giving body response, muscle tensions, blood sugar, pressure, immune system changes. Anxiety or aggression, dependant on the individuals reaction to the stress in the body, which depends on the situation. Amygdalas reaction to what signals are send in different situations can help depression and anxiety reactions.
%	\cite{Perspectives in the Management of Anxiety Disorders} \cite{Central Command Neurons of the Sympathetic Nervous System: Basis of the Fight-or-Flight Response}
	
Anxiety and aggression
	The physiological changes in the body can induce anxiety and aggression, this depends on the amygdala, as the information send is dependant on how the amygdala has been previous stimulated. If in a current state of constant stress, the amygdala might become overstimulated, sending strongly biased information about the perception of the current situation easily creating anxiety. %	\cite{How you perceive threat determines your behavior}
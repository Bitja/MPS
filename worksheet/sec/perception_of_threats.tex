\section{Perception of Threats} \label{perception}
Animals as well as humans have a build in sensor system to perceive threats to their survival. The sensor system uses the senses of the body as well as the mind in its aspect of memorizing a similar situation and the evaluation of the magnitude of the threat. We humans have through the years developed a more detailed sensor system, and we are able to use it, not only to perceive situations threatening our survival, but also recognize situations which might compromise our social status. The most common way of experiencing the evolved sensor system, is when being put in a stressful situation, as this would activate the autonomic nervous system in a physiological reaction. A scenario could be forgetting to deliver a report at work and be pointed out by your boss, or just being stressed in general due to the current workload.


\subsection{Changes in Behaviour due to Perception}
It is believed that emotions have been developed to affect our behaviour when put in a stressful situation, in order to identify and react to a perceived threat, to increase the survival rate. According to \cite{threat_behaviour}, emotions can change our behaviour in a positive or negative direction, depending on the situation. This also gives us the opportunity to prepare for stressful situations by training the control of our emotional response to a threat. A reaction to the current situation can this way be altered by past event, which could be the aforementioned training of emotional control or recall of a similar situation. As the action or motor performed by the individual is directly linked to the emotional state, which is dependant on our past memories or physiological reaction. The reaction time in humans is directly related to how we perceive an emerging threat. If the threat is pointed towards us: If a gun is pointed at the viewer, our reaction time is slowed. When a gun is pointed towards another person, our reaction time is increased compared to a neutral state. This gives evidence of our perception of the threat changes how the body reacts to the situation.
\cite{threat_behaviour}
%Mutilation pictures induce a freeze like response. Freeze, and into flight, and if not possible to flee, fight.

\subsection{Bodily Changes as Result of Threat Perception}
When a threat is perceived, the first reaction of the brain is to activate the amygdala part, which processes emotions, memory and decision-making. The information about the situation is send to hypatalamus, which creates Adrenocorticotropt hormon (ACTH). This starts the production of the stress hormone cortisol and adrenaline, making the body respond by increasing heart rate, blood pressure, decrease the immune system functionality and digestion also making a constant stressful environment harmful to the health. If the amygdala is constantly negatively stimulated, it can lead to depression and other anxiety, altering the reaction to stressful or harmful situations. Depression can in this sense also be helped by changing the way the amygdala reacts to stressful situations as it has a direct relation to the information send further to the autonomic nervous system (ANS) regulating the autonomic body functions. One of the primary parts of the ANS is the sympathetic nervous system, which is sometimes referred to as the fight or flight system, as it automatically alters the autonomic body functions to create a sense of anxiety or aggression. The anxiety or aggression creates a bias for what action is made in a stressful situation, and the next section will further dig into the action taken depending on the physiological changes in the body when put into a stressful situation.\cite{human_physiology} \cite{perspective_anxiety} \cite{CNS}
%	\cite{How you perceive threat determines your behavior}
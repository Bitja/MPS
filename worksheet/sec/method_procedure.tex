With this knowledge about different stress reaction patterns, a test scenario was constructed to examine the following problem statement.

“In the game situation a stress response can be provoked for the user, and a response can be grouped into one of the four categories of fight, flight, freeze and tend/befriend.”

\section{Methods}
A scenario was constructed in which a stress response was provoked from the user. A pair of test participants were competing against each other in a tablet game (see Appendix). They went into the test believing they were competing fairly, but in truth the final level of the second test participant was rigged, so that he or she could not complete the level. The subject’s reaction to the stress situation of being unable to complete the level, was then observed by looking at a preset preliminaries(see Appendix), and grouped into one of the pre-defined patterns of reaction fight, flight, freeze, and tend/befriend. For the test there were two cameras, one focusing on the tablet and one capturing the two participants’ facial expressions, see the setup in Appendix Figure ??. 

\section{Participants}
When choosing test participants we went for pairs of students already sitting in groups and made the assumption, that they would be acquaintances, and more inclined to engage in competition, than two strangers. We also chose to test on different combinations of males and females, as it is illustrated on table (see Appendix Table \ref{tab: pairs}). All the subjects were 3rd semester Medialogy students at Aalborg University and had the following demographics(See Appendix). 

\section{Procedure}
The procedure for the experiment was controlled by the test conductor, who instructed the test participants what to do, while another test conductor was in charge of the camera setup. When the test participants entered the room, they were asked to give verbal consent to being recorded. After the participants received the instructions, see Appendix Section ??, the test conductors left the test room, so the response of the participants was not affected by the presence of test conductors. Subject A played through the three levels of the tablet game, and after each level the score is noted down on a scoreboard by the participants themselves, in order for them to keep track on who was in the lead. After completing the third level they switched seats, and Subject B played through what appeared to be three similar levels, as they were unaware of the third level being rigged. The game was completed, after the timer ran out in the broken level, and the participants gestured to the test conductors that they were done. After the test session followed a semi-structured interview of both participants (See Appendix). If they did not approach the subject of the broken level themselves, the test conductor revealed this to them and got their final comments. 

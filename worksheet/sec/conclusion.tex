\section{Conclusion}
\begin{itemize}
\item When a person perceives a stressful situation, the autonomic nervous system is  activated as a physiological reaction.
\item The human response to stress can be categorised as fight, flight, freeze, tend/befriend.
\begin{itemize}
\item Fight is an active/aggressive reaction
\item Flight is an active/non-aggressive reaction
\item Freeze is a passive reaction 
\item Tend/befriend is an active/social reaction
\end{itemize}
\end{itemize}

In the experience we managed to provoke all of the four different stress reactions, and in some instances, they came across as a mix of reactions. The reactions would start in one category and then transition to other categories as time passed. With the sample size being so small, however, it was not possible to significantly conclude that the different combination of gender had any influence on the provoked reaction.

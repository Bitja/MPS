\section{Motors/Actions}
The last step of the human reaction to a stressful situation, is the motor or action part. The model can be referred to as the human information processing model, which consists of four steps:
\begin{itemize}
\item Sensing
\item Perception
\item Cognition
\item Motor (Acting)
\end{itemize}
The motor steps is the actions we can start interpreting as how the full processing of the perceived information has occurred. We define the different motor categories as fight, flight, freeze and tend/befriend, and in the following we define each category. To provoke a stressed reaction we use a game which has a level that can not be completed to create a stressful situation for the individual. The game used for the test is described in a later section. The categories are described with a definition of the action and what information processing the user has gone through to react according to the category. The specific actions related to the category will be described, as this will be the basis of how we can categorise the stressful reaction to the broken level in the game. \cite{humanprocessor}

\subsection{Fight}
The fight action is greatly biased by an aggressive state, which is shown in actively trying to deal with a posing threat. The feeling of aggression is associated with the will to fight, which is induced by the physiological response of releasing hormones when a threat is perceived. As adrenaline is released, heart rate rises, muscles tense up, making the body ready to fight of the threat. This motor is part of the body's automatic energy conserving system, where the body can be in a state of lowest possible energy until a threat is perceived, where it in an instant can release the hormones suddenly raising the energy available to a much higher level than normal. Fighting can be seen as tension building up in the body, needed to be expressed physically.

The fight reaction to a stressed situation is becoming more and more active in solving the problem or fighting of the threat, which can result in a physical reaction, violence, verbal expression of anger or annoyance.

\subsection{Flight}
The flight response can be associated with the feeling of anxiety, where fleeing is considered the best option. Flight is often the first action which is considered as it would raise the possibility of survival compared to getting into a fight. If escaping the situation is not an option, the fight response is initialised through the increasing emotion of anger. The flight response is as fight also amplified by the hormone release happening in a threatening or stressful situation. Muscles tense and heart rate increases to help a faster escape. For the flight response, blood will gather in the legs compared to the fight response, where blood flows to the upper body, either giving extra physical strength to escaping or fighting the situation. A terminal depression is often associated with constant anxiety, because the individual is more biased towards the flight response than fight response, creating anxiety in all mildly stressful situations.

Flight is represented fleeing or escaping a situation because of rising anxiety and the idea of not being able to solve the problem or deal with the posing threat. It is often seen as fleeing the location or removing one self from the situation.

\subsection{Freeze}
As a last resort, freeze can be used when escaping and fighting is not a possible solution. The best example is when being attacked by a bear, where running away would increase the chance of it chasing and the physical superiority of the bear would omit the fighting response. Playing dead or becoming passive would give the greatest chance of survival, making the freeze response become the best option. Freeze can be triggered when no obvious solution is present, either in fight or flight, but might transition into the other motors as a solution presents itself by evaluation of the situation. Freeze can be trigger when muscle tense up, but no solution in fight or flight can be chosen to deal with the threat or situation.

The freeze response can be seen as becoming passive, trying the same solution over and over can be seen as passiveness if it does not help remove the stressful situation, but done because no other response is possible.

\subsection{The Reactions According to Tend/Befriend}
In a study done by \cite{tendbefriend}(Taylor,2000) it is proposed that a female stress response can be characterized by the the pattern tend and befriend. This pattern involves joining and strengthening social groups in order to share resources, in particular in groups of other females. The study suggests that this pattern builds on biobehavioral attachment/caregiving system that depends on oxytocin, estrogen, and endogenous opioid mechanisms, among other neuroendocrine underpinnings, and that this is an alternative to the biobehavioral response of fight/flight. In this the actor is actively trying to solve a stressful situation by strengthening social networks and relying on shared resources. \cite{tendbefriend}

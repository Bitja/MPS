\section{Motors/Actions}
The last step of the human reaction to a stressful situation, is the motor or action part. The model can be referred to as the human information processing model, which consists of four steps:
\begin{itemize}
\item Sensing
\item Perception
\item Cognition
\item Motor (Acting)
\end{itemize}
The motor steps is the actions we can start interpreting as how the full processing of the perceived information has occurred. We define the different motor categories as fight, flight, freeze and tend/befriend, and in the following we define each category. To provoke a stressed reaction we use a game which has a level that can not be completed to create a stressful situation for the individual. The game used for the test is described in a later section. The categories are described with a definition of the action and what information processing the user has gone through to react according to the category. The specific actions related to the category will be described, as this will be the basis of how we can categorise the stressful reaction to the broken level in the game. %\cite{http://www2.parc.com/istl/groups/uir/publications/items/UIR-1986-05-Card.pdf}

\subsection{Fight}
Fight and how it is represented in the anger emotion

\subsection{Flight}
Flight represented in the anxiety emotion

\subsection{Freeze}
Freeze in giving up hope in the situation